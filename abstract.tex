\chapter*{Abstract}
Cloud Manufacturing is a resource-sharing paradigm that provides on- demand access to a pool of manufacturing resources and capabilities to utilize geographically scattered resources in a service-oriented model. These services are rapidly provisioned and released with minimal management effort via the Industrial Internet of Things and its underlying IT infrastructure, architecture models, and data and information exchange protocols and standards. In this context, the tradeoff between resources’ autonomy and independence exigencies and platform needs for centralized control and coordination is a crucial enabler factor for implementing such vertically or horizontally integrated cyber-physical systems for intelligent manufacturing. The introduction of resources autonomy and network independence in a distributed cloud manufacturing system enables platforms with equal and open access to shared resources in a more sustainable way and potentially with higher scalability of manufacturing resources and capabilities.\\
This work aims to develop a framework to manage distributed operations in cloud manufacturing based on autonomous resources. This research investigates network architectures in the context of distributed Cloud Manufacturing systems with autonomous and independent resources to identify critical parameters that determine whether an efficient deployment is viable for a given scenario.\\
The framework includes: (i) a network architecture for a distributed Cloud Manufacturing platform based on autonomous nodes; (ii) a Multi-agent Systems architecture for managing communications and coordination issues in distributed operations; (iii) an implementation of the proposed network architecture in the context of large Additive Manufacturing networks; (iv) a unique optimization algorithm that combines scheduling and logistics issues inside such network. Additionally, an implementation of the Multi- Agent Systems architecture has been developed to offer practical guidance for implementing the framework into context closer to the industry and real life.\\
A literature review was conducted to analyze the research area to accomplish the goal and objectives of this work. Next, a framework was outlined to identify, assess, and control dynamics and issues inside the network. Two well-known and established approaches were used to implement the communication and coordination system and the optimization of the platform in this research: Multi-agent Systems to tackle the dynamic task arrival, the downtime of machines, the identification of the anomalous tasks; and Operation Research techniques to tackle logistics and to schedule global optimization for a job order.\\
Results from this work are beneficial for both academia and industry in understanding aspects involving new varieties of cloud manufacturing networks. The principal contribution is a framework that offers new insights and outlines new issues on how to deal with autonomous and independent resources inside a Cloud Manufacturing platform and how to manage global optimization and long-term sustainability of such networks. Finally, this study also introduced a novel cloud manufacturing taxonomy, including a list of actors, a list of platform services and functionalities.
